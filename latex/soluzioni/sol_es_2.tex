\documentclass[11pt,a4]{article}

\usepackage[margin=2cm]{geometry}

\usepackage{amsmath}
\usepackage{url}

\usepackage[utf8]{inputenc}

% Default fixed font does not support bold face
\DeclareFixedFont{\ttb}{T1}{txtt}{bx}{n}{10} % for bold
\DeclareFixedFont{\ttm}{T1}{txtt}{m}{n}{10}  % for normal

% Custom colors
\usepackage{color}
\definecolor{deepblue}{rgb}{0,0,0.5}
\definecolor{deepred}{rgb}{0.6,0,0}
\definecolor{deepgreen}{rgb}{0,0.5,0}

\usepackage{listings}

% Python style for highlighting
\newcommand\pythonstyle{\lstset{
language=Python,
numbers=left,
numberstyle=\tiny\color{black},
basicstyle=\ttm,
otherkeywords={self},             % Add keywords here
keywordstyle=\ttb\color{deepblue},
emph={MyClass,__init__},          % Custom highlighting
emphstyle=\ttb\color{deepred},    % Custom highlighting style
stringstyle=\color{deepgreen},
frame=tb,                         % Any extra options here
showstringspaces=false            % 
}}


% Python environment
\lstnewenvironment{python}[1][]
{
\pythonstyle
\lstset{#1}
}
{}

% Python for external files
\newcommand\pythonexternal[2][]{{
\pythonstyle
\lstinputlisting[#1]{#2}}}

% Python for inline
\newcommand\pythoninline[1]{{\pythonstyle\lstinline!#1!}}


\usepackage{collectbox}

\newcommand{\mybox}[2]{$\quad$\fbox{
\begin{minipage}{#1cm}
\hfill\vspace{#2cm}
\end{minipage}
}}

\usepackage{fancyhdr}
\pagestyle{fancy}
\rhead{Programmazione 1 - Esercitazione 2}

\usepackage[T1]{fontenc}
\usepackage[utf8]{inputenc}
\usepackage{lmodern}
%%%%%%%%%%%%%%%%%%%%%%%%%%%%%%%%%%%%%%%%%%%%%%%%%%%%%%%%%
% Source: http://en.wikibooks.org/wiki/LaTeX/Hyperlinks %
%%%%%%%%%%%%%%%%%%%%%%%%%%%%%%%%%%%%%%%%%%%%%%%%%%%%%%%%%
\usepackage{hyperref}
\usepackage{graphicx}
\usepackage[english]{babel}

\usepackage{bm}
\usepackage{amsmath}
\usepackage{amsfonts}

\usepackage{amsthm}
\newtheorem{definition}{Definizione}
\newtheorem{theorem}{Teorema}
\renewcommand*{\proofname}{Dimostrazione}
\newtheorem{example}{Esempio}
\newtheorem{lemma}{Lemma}
\newtheorem{exercise}{Esercizio}
\newtheorem{property}{Proprietà}

\usepackage[ruled,vlined,linesnumbered]{algorithm2e}

\newcommand{\xstar}{x^*}
\newcommand{\bxstar}{\bm{x^*}}
\newcommand{\bx}{\bm{x}}
\newcommand{\Rn}{\mathbb{R}^n}
\newcommand{\RR}{\mathbb{R}}
\newcommand{\norm}[1]{\left\lvert \left\lvert #1 \right\lvert \right\lvert}

\newcommand{\fx}{f(x)}

\newcommand{\gradfx}{\nabla \fx}
\newcommand{\Gx}{\nabla f(x)}
\newcommand{\Gk}{\nabla f(x_k)}
\newcommand{\Gs}{\nabla f(\xstar)}

\newcommand{\Hx}{\nabla^2 f(x)}
\newcommand{\Hk}{\nabla^2 f(x_k)}
\newcommand{\Hs}{\nabla^2 f(\xstar)}
\newcommand{\hess}{\nabla^2 f}

\newcommand{\step}{\alpha}
\newcommand{\Seqx}{\{ x_k \}}

\usepackage{mathtools}
\newcommand\myeq{\stackrel{\mathclap{\normalfont\mbox{def}}}{=}}

\usepackage{listings}
\lstset
{ 
    language=Matlab,
    basicstyle=\normalsize,
    numbers=left,
    stepnumber=1,
    showstringspaces=false,
    tabsize=1,
    breaklines=true,
    breakatwhitespace=false,
   frame=single
}


\begin{document}
\thispagestyle{empty}
\hrule
\begin{center}
   {\Large {\bf Programmazione 1 \hspace{3cm} $\quad \quad$ Soluzioni Esercitazione 2}}
\end{center}
\hrule

\section*{}
Le soluzioni sono anche fornite in un unico script python sul sito: 
\begin{center}
	\url{https://github.com/mathcoding/programming/tree/master/scripts}
\end{center}

\begin{enumerate}
%%%%%%%%%%%%%%%%%%%%%%%%%%%%%%%%%%%%%%%%%%%%%%%%%%%%%%%%%%%%%%%%%%%%%%%%%%%%%
\item Scrivere una procedura che calcoli l'ennesimo numero di Fibonacci usando un {\bf processo iterativo}
(si veda il {\tt Lab 4} per la definizione di processo iterativo).
\begin{python}
def FibRec(n):
    """ Calcolo di Fibonacci con PROCESSO RICORSIVO """
    if n == 0 or n == 1:
        return n
    return FibRec(n-1) + FibRec(n-2)

def FibIter(n):
    """ Calcolo di Fibonacci con PROCESSO ITERATIVO """
    def FibI(a, b, counter):
        if counter < 2:
            return a
        return FibI(a+b, a, counter-1)
    
    return FibI(1, 0, n)
\end{python}

%%%%%%%%%%%%%%%%%%%%%%%%%%%%%%%%%%%%%%%%%%%%%%%%%%%%%%%%%%%%%%%%%%%%%%%%%%%%%
\item Una funzione $f$ è definita dalla regola seguente:
\begin{align}
	f(n) = \left\{\begin{array}{ll} n & \mbox{if } n<3 \\ f(n-1)+2\,f(n-2)+3\,f(n-3) & \mbox{if } n \geq 3 \end{array} \right.
\end{align}
\begin{enumerate}
\item Si scriva una procedura che calcoli $f$ usando un {\bf processo ricorsivo}.
\item Si scriva una procedura che calcoli $f$ usando un {\bf processo iterativo}.
\end{enumerate}
\begin{python}
def FnRec(n):
    """ Calcolo di f(n)  = f(n-1)+2*f(n-2)+3*f(n-3), con f(0)=0, f(1)=1, f(2)=2 
        (uso di processo ricorsivo) """
    if n < 3:
        return n
    return FnRec(n-1) + 2*FnRec(n-2) + 3*FnRec(n-3)

def FnIter(n):
    """ Calcolo di f(n)  = f(n-1)+2*f(n-2)+3*f(n-3), con f(0)=0, f(1)=1, f(2)=2 
        (uso di processo iterativo) """
    def FnI(a, b, c, counter):
        if counter < 3:
            return a
        return FnI(a+2*b+3*c, a, b, counter-1)
    
    return FnI(2, 1, 0, n)
\end{python}

%%%%%%%%%%%%%%%%%%%%%%%%%%%%%%%%%%%%%%%%%%%%%%%%%%%%%%%%%%%%%%%%%%%%%%%%%%%%%
\item Si scrive un predicato che ci dica se un dato numero $n$ sia un numero primo.
Si può usare la definizione che $n$ è un numero primo, se e solo se $n$ è il suo più piccolo divisore maggiore di uno
(suggerimento: scrivere prima una funzione che trova il più piccolo divisore di $n$).

È possibile scrivere questa procedura in modo tale che l'ordine di crescita per il numero di operazioni richieste sia $\Theta(\sqrt{n})$?
\begin{python}
def IsPrime(n):
    """ Predicato che restituisce "True" quando "n" e` un numero primo """
    def MinorDivisore(a, n):
        if a*a > n:
            return n
        if n % a == 0:
            return a
        return MinorDivisore(a+1, n)
    
    if n == 0:
        return False
    if n == 1:
        return True
    return MinorDivisore(2,n) == n
\end{python}
 
%%%%%%%%%%%%%%%%%%%%%%%%%%%%%%%%%%%%%%%%%%%%%%%%%%%%%%%%%%%%%%%%%%%%%%%%%%%%%
\item La procedura {\tt Sommatoria} vista nel {\tt Lab 7} genera un processo ricorsivo lineare.
La stessa procedura può essere riscritta in modo tale che il processo generato sia iterativo:
scrivere la procedura che genera un processo iterativo lineare.
\begin{python}
def SommatoriaIter(F, a, Next, b):
    def SommatoriaI(a, b, accumulator):
        if a > b:
            return accumulator
        return SommatoriaI(Next(a), b, accumulator+F(a))
    
    return SommatoriaI(a, b, 0)    
def IntegraleIter(F, a, b, dx):
    def AddDx(x):
        return x + dx
    return dx*SommatoriaIter(F, a+dx/2, AddDx, b)
\end{python}

%%%%%%%%%%%%%%%%%%%%%%%%%%%%%%%%%%%%%%%%%%%%%%%%%%%%%%%%%%%%%%%%%%%%%%%%%%%%%
\item In maniera analoga alla funzione {\tt Sommatoria}, si può definire una funzione {\tt Produttoria}
che restituisce i valori dei prodotti di una funzione valutata in un insieme di punti definiti da un dato intervallo $[a,b]$:
$$
	\prod_{n=a}^b f(n) = f(a)\cdot ... \cdot f(b)
$$
Scrivere tale funzione e mostrare come sia possibile usarla per calcolare $n!$
\begin{python}
def ProduttoriaRec(F, a, Next, b):
    if a > b:
        return 1
    else:
        return F(a) * ProduttoriaRec(F, Next(a), Next, b)
print("Es. 2.5 => ProduttoriaRec(1, 5): ", ProduttoriaRec(lambda x: x, 1, Inc, 5))
\end{python}

%%%%%%%%%%%%%%%%%%%%%%%%%%%%%%%%%%%%%%%%%%%%%%%%%%%%%%%%%%%%%%%%%%%%%%%%%%%%% 
\item Eseguire il grafico sovrapposto delle funzioni seguenti: 
$$R(n)=n, R(n)=10^5\,n, R(n)=n^2, R(n)=\log{n}, R(n)=n\log{n}, R(n)=1.6^n, R(n)=2^n.$$
\noindent {\bf ATTENZIONE:} Questa procedura potrebbe non funzionare correttamente all'interno di Spyder.
Tuttavia uno script python può essere eseguito anche da linea di comando, come visto in laboratorio.
Per i grafici, si consiglia di vedere pochi grafici alla volta con velocità di crescita simili.

\begin{python}
from numpy import linspace
from matplotlib.pyplot import plot, xlabel, ylabel, show, legend, clf
from math import log
    
def PlotFunzioni():   
    D = linspace(1, 10, 100)
    clf()
    #plot(D, [x for x in D], label="y=x")
    plot(D, [x*x for x in D], label="y=x^2")
    plot(D, [log(x) for x in D], label="y=log(x)")
    plot(D, [x*log(x) for x in D], label="y=x*log(x)")
    
    #D = linspace(1, 10, 100)
    #plot(D, [1.6**x for x in D], label="y=1.6^x")
    #plot(D, [2**x for x in D], label="y=2^x")
    xlabel("x")
    ylabel("y=f(x)")
    legend(loc="upper left", shadow=True)
    show()
\end{python}


\end{enumerate}

\end{document}
