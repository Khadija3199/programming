\documentclass[11pt,a4]{article}

\usepackage[margin=2cm]{geometry}

\usepackage{multicol}

\usepackage{amsmath}
\usepackage{url}

\usepackage{amssymb}

\usepackage{amsmath}
\usepackage{url}

\usepackage[utf8]{inputenc}

% Default fixed font does not support bold face
\DeclareFixedFont{\ttb}{T1}{txtt}{bx}{n}{10} % for bold
\DeclareFixedFont{\ttm}{T1}{txtt}{m}{n}{10}  % for normal

% Custom colors
\usepackage{color}
\definecolor{deepblue}{rgb}{0,0,0.5}
\definecolor{deepred}{rgb}{0.6,0,0}
\definecolor{deepgreen}{rgb}{0,0.5,0}

\usepackage{listings}

% Python style for highlighting
\newcommand\pythonstyle{\lstset{
language=Python,
basicstyle=\ttm,
otherkeywords={self},             % Add keywords here
keywordstyle=\ttb\color{deepblue},
emph={MyClass,__init__},          % Custom highlighting
emphstyle=\ttb\color{deepred},    % Custom highlighting style
stringstyle=\color{deepgreen},
frame=tb,                         % Any extra options here
showstringspaces=false            % 
}}


% Python environment
\lstnewenvironment{python}[1][]
{
\pythonstyle
\lstset{#1}
}
{}

% Python for external files
\newcommand\pythonexternal[2][]{{
\pythonstyle
\lstinputlisting[#1]{#2}}}

% Python for inline
\newcommand\pythoninline[1]{{\pythonstyle\lstinline!#1!}}


\usepackage{collectbox}

\newcommand{\mybox}[2]{$\quad$\fbox{
\begin{minipage}{#1cm}
\hfill\vspace{#2cm}
\end{minipage}
}}


\usepackage{fancyhdr}
\pagestyle{fancy}
\rhead{Programmazione 1 - Tutorato 3}

\usepackage[T1]{fontenc}
\usepackage[utf8]{inputenc}
\usepackage{lmodern}
%%%%%%%%%%%%%%%%%%%%%%%%%%%%%%%%%%%%%%%%%%%%%%%%%%%%%%%%%
% Source: http://en.wikibooks.org/wiki/LaTeX/Hyperlinks %
%%%%%%%%%%%%%%%%%%%%%%%%%%%%%%%%%%%%%%%%%%%%%%%%%%%%%%%%%
\usepackage{hyperref}
\usepackage{graphicx}
\usepackage[english]{babel}

\usepackage{bm}
\usepackage{amsmath}
\usepackage{amsfonts}

\usepackage{amsthm}
\newtheorem{definition}{Definizione}
\newtheorem{theorem}{Teorema}
\renewcommand*{\proofname}{Dimostrazione}
\newtheorem{example}{Esempio}
\newtheorem{lemma}{Lemma}
\newtheorem{exercise}{Esercizio}
\newtheorem{property}{Proprietà}

\usepackage[ruled,vlined,linesnumbered]{algorithm2e}

\newcommand{\xstar}{x^*}
\newcommand{\bxstar}{\bm{x^*}}
\newcommand{\bx}{\bm{x}}
\newcommand{\Rn}{\mathbb{R}^n}
\newcommand{\RR}{\mathbb{R}}
\newcommand{\norm}[1]{\left\lvert \left\lvert #1 \right\lvert \right\lvert}

\newcommand{\fx}{f(x)}

\newcommand{\gradfx}{\nabla \fx}
\newcommand{\Gx}{\nabla f(x)}
\newcommand{\Gk}{\nabla f(x_k)}
\newcommand{\Gs}{\nabla f(\xstar)}

\newcommand{\Hx}{\nabla^2 f(x)}
\newcommand{\Hk}{\nabla^2 f(x_k)}
\newcommand{\Hs}{\nabla^2 f(\xstar)}
\newcommand{\hess}{\nabla^2 f}

\newcommand{\step}{\alpha}
\newcommand{\Seqx}{\{ x_k \}}

\usepackage{mathtools}
\newcommand\myeq{\stackrel{\mathclap{\normalfont\mbox{def}}}{=}}

\usepackage{listings}
\lstset
{ 
    language=Matlab,
    basicstyle=\normalsize,
    numbers=left,
    stepnumber=1,
    showstringspaces=false,
    tabsize=1,
    breaklines=true,
    breakatwhitespace=false,
   frame=single
}


\begin{document}
\thispagestyle{empty}
\hrule
\begin{center}
   {\Large {\bf Programmazione 1 \hspace{3cm} $\quad \quad \quad$ Tutorato 3}}
\end{center}

\hrule

%%%%%%%%%%%%%%%%%%%%%%%%%%%%%%%%%%%%%%%%%%%%%%%%%%%%%%%%%%%%%%%%%%%%%%%%%%%%%
\section*{}

\begin{enumerate}

%%%%%%%%%%%%%%%%%%%%%%%%%%%%%%%%%%%%%%%%%%%%%%%%%%%%%%%%%%%%%%%%%%%%%%%%%%%%%
\item Scrivere una funzione che prende in input una stringa e un dizionario. Se il dizionario è vuoto restituisce una stringa criptata e un dizionario che decripta, altrimenti utilizza il dizionario dato in input per decriptare la stringa.


%%%%%%%%%%%%%%%%%%%%%%%%%%%%%%%%%%%%%%%%%%%%%%%%%%%%%%%%%%%%%%%%%%%%%%%%%%%%%
\item Scrivere una funzione che dato $n$ restituisce la matrice di Hilbert di ordine $n$:

\begin{center}
$H=\begin{pmatrix}
       1      & \frac{1}{2} & \frac{1}{3} & \cdots \\[1ex] 
  \frac{1}{2} & \frac{1}{3} & \frac{1}{4} &  \\[1ex]
  \frac{1}{3} & \frac{1}{4} & \frac{1}{5} &  \\
    \vdots    &             &             & \ddots 
\end{pmatrix}$
$H(i,j)=\frac{1}{i+j+1}$ $i,j=0,\dots,n-1$
\end{center}

%%%%%%%%%%%%%%%%%%%%%%%%%%%%%%%%%%%%%%%%%%%%%%%%%%%%%%%%%%%%%%%%%%%%%%%%%%%%%
\item Scrivere una funzione {\tt MapMat(Matrix, f)} che implementa la "map" su matrici intese come liste di liste. 
Ad esempio:\\
\begin{center}
$
\left(
\begin{bmatrix} 
0 & \pi & 2\pi\\[4pt]
\pi & \frac{3}{2}\pi & \frac{\pi}{4} \\[4pt]
\pi & 0 & \frac{-1}{4}\pi\end{bmatrix}, sin \right) \longmapsto 
\begin{bmatrix} 
0 & 0 & 0\\[4pt]
0 & -1 & \sqrt[]{2}/2 \\[4pt]
0 & 0 & -\sqrt[]{2}/2\end{bmatrix}
$
\end{center}

%%%%%%%%%%%%%%%%%%%%%%%%%%%%%%%%%%%%%%%%%%%%%%%%%%%%%%%%%%%%%%%%%%%%%%%%%%%%%
\item Scrivere una funzione che data una matrice e un predicato $p$, 
restituisce la matrice stessa portando a 0 gli elementi $x$ tali che $p(x)=False$.\\

\item Ricordando le due funzioni:
\begin{python}
def MakeImage(F, n, scale=0.01):
    data = [scale*i for i in range(-n,n)]
    return np.matrix([[F(complex(a, b)) for a in data] for b in data])
\end{python}
\begin{python}
def DrawImage(F, n, scale):
    m = MakeImage(F, n, scale)
    plt.figure(figsize=(8,8))
    img = plt.imshow(m, extent=(-scale*n, scale*n, -scale*n, scale*n), cmap='hot')
    plt.show()
\end{python}
implementare una funzione che prende in input una funzione $f:\mathbb{C}\rightarrow \mathbb{C}$ e restituisce il grafico dell'insieme $J(f)=\{ z\in \mathbb{C}: \exists \delta$ tale che $|(f^n(z))|<\delta$ $\forall n \}$ dove:
$f^n=\underbrace{f\circ\ldots\circ f}_{n\ \mathrm{volte}}$
\end{enumerate}

\end{document}